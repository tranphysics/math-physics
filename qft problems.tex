\documentclass[10pt]{scrartcl}
\usepackage[sexy]{evan}
\usepackage{braket}
\usepackage{color}   %May be necessary if you want to color links
\usepackage{hyperref}
\DeclareOldFontCommand{\bf}{\normalfont\bfseries}{\mathbf}

\hypersetup{
    colorlinks=true, %set true if you want colored links
    linktoc=all,     %set to all if you want both sections and subsections linked
    linkcolor=blue,  %choose some color if you want links to stand out
}
\usepackage{geometry}
    %\usepackage{showframe} %This line can be used to clearly show the new margins

\newgeometry{vmargin={30mm}, hmargin={20mm,30mm}}   % set the margins
\begin{document}
\title{Notes 2018, Part I} % Beginner
\date{December 2017}
\maketitle

\begin{abstract}
	\sffamily\small
	Here is a collected notes on physics readings.
\end{abstract}

\vspace{1em}
%%fakesection Notations + Acknow
\tableofcontents
\newpage


\section{Problems}

\subsection{Peskin}
\subsubsection{Peskin Problem 11.1}

\paragraph{Part a}
\begin{lemma}
For a free field $\phi(x)$, the following holds:
$$\braket{ e^{i \phi(x)} e^{-i \phi(0)}} = e^{D(x)-D(0)}$$
where $ D(x) \equiv \braket{\phi(x-y) \phi(y)}_c$
\end{lemma}

\begin{proof}
Define Z(J) as:
\begin{align}
Z(J) = \int d \phi \exp\left(i \int d^d x \phi D^{-1} \phi + J \phi) \right)
\end{align}
We prove by evaluating the time ordered correlator directly in the path integral:
\begin{align}
\braket{e^{i (\phi(x)- \phi(0))}} & = {\int d \phi \exp\left(i \int d^d x \phi D^{-1} \phi + i (\phi(x)- \phi(0)) \right) \over Z(0)} \\
 & = {Z(J_0) \over Z(0)} \\
 & = e^{i\frac12 J_0 D J_0} \text{ (Formula for gaussian integrals)}
\end{align}
We note the equation works if we define $J_0(y) = \delta(x-y)- \delta(y)$.
This is just a gaussian integral, with result being:
\begin{align}
\braket{e^{i (\phi(x)- \phi(0))}} &= \exp \left(i \frac12  \int dx' \int dy' (\delta(x-x')-\delta(x')) D (\delta(x-y') -\delta(y'))\right) 
\end{align}
There are 2 terms for each propagator which don't vanish, at index x and 0, and they cancel the $\frac12$:
\begin{align}
\braket{e^{i (\phi(x)- \phi(0))}} &= \exp \left(i \left( D(x)- D(0) \right) \right) 
\end{align}
\end{proof}

\paragraph{Part b}
The most general term is of the form:
\begin{align}
\phi^n (\partial \phi)^(2m)
\end{align}

$n = 0$ because of U(1) symmetry.  The remaining term has the following coupling dimensions:
\begin{align}
\underbrace{[g]}_{\text{coupling dimension}} + 2m \left( [\partial] + [\phi] \right) & = d \\
g &= d- 2m \left( [\partial] + [\phi] \right) \geq 0 \\
[\partial]  &= 1 \\
[\phi] &= {d - 2 \over 2}  \text{ (Kinetic term is canonically normalized) } \\
m & \leq {d \over 2 (1 + {d-2 \over 2})} \\
m & \leq 1
\end{align}

This implies the most general renormalizable coupling is of the form:

\begin{align}
\rho (\partial \phi)^2
\end{align}

\paragraph{Part c}
This is just coulomb potential in d dimensions:
\begin{align}
D(k^2) = {1 \over \rho k^2}  \\
\rho \nabla^2 D(x) =  \delta(x) \\
\end{align} 
What the previous equation states is that D(x) satisfies gauss's law (it solves poisson's equation),
and we can therefore immediately obtain the potential.

We get that the following cases:

\begin{align}
\begin{cases}
D(x) = {1 \over \rho S_d r^{d-2}} \text{ ($d > 2$, $S_d$ is the surface of a unit d dimensional sphere)} \\
D(x) = { \ln(r) \over \rho S_d} \text{ (d = 2 )}
\end{cases}
\end{align}


The spin spin correlation is obtained from part a:

\begin{align}
\begin{cases}
D(x) \propto e^{1 \over \rho x^{d-2}}  \text{ ( $d > 2$)} \\
D(x) \propto {1 \over \rho x} \text{ (d = 2)} \\
D(x) \propto e^{-x} \text{(d = 1)} 
\end{cases}
\end{align}
We see that there is long range order in dimensions 3, 4 and greater.  At dimension 2, we have algebraic order (conformal behavior).
In dimensions 1, we have loss of long range order.

\end{document}